% 使用 UTF-8 编码 
\documentclass{ctexart}
\usepackage{listings}
\usepackage{enumitem}
\usepackage{courier}
\usepackage{booktabs}
\setCJKmainfont{SimSun}
\setlength\parindent{0em}

\lstset{basicstyle=\footnotesize\ttfamily,breaklines=true}

\author{程振}
\title{高级并行程序作业——MPI}

\begin{document}
\maketitle

1. \lstinline{MPI_Wtime} 计时
编译并且运行下面的\lstinline{MPI_Wtime}样例程序,了解使用\lstinline{MPI_Wtime}计时的方法。

timer.c
\lstset{language=C}
\begin{lstlisting}
  #include "mpi.h"
  #include <stdio.h>
  int main(int argc, char* argv[])
  {
      int rank, size;
      double t_start, t_end, l_time, g_time;
      MPI_Init(&argc,&argv);
      MPI_Comm_rank(MPI_COMM_WORLD, &rank);
      MPI_Comm_size(MPI_COMM_WORLD, &size);
      t_start = MPI_Wtime();
      sleep(2);
      if (rank == 1) sleep(1);
      t_end= MPI_Wtime();
      l_time = t_end - t_start;
      g_time = 0;
      MPI_Reduce(&l_time, &g_time, 1, MPI_DOUBLE, MPI_MAX, 0, MPI_COMM_WORLD);
      if (!rank) 
      printf("Local Time is %f, Global Time is %f\n", l_time, g_time);
      MPI_Finalize();
      return 0;
  }
\end{lstlisting}

编译命令:\lstinline{mpicc timer.c –o timer}

运行命令(TH2):\lstinline{yhrun –n 4 -p free ./timer}

运行命令(Local):\lstinline{mpirun –n 4 -p free ./timer}

运行结果(TH2):Local Time is 2.000131, Global Time is 3.000175

运行结果(Local):Local Time is 2.003052, Global Time is 3.006091

说明:所有进程执行的操作有sleep(2), 其中1号进程额外执行了sleep(1),
最后在0号进程进行规约操作,获取的local time为0号进程的执行时间,即2s,
global time为1号进程的时间(MAX),即3s。
\\

2. 替换timer.c中的sleep语句为一段计算程序,并为其计时。
\\

说明:使用计算$\pi$值的程序替换1号进程中的sleep语句。
源程序见ex2.c。

运行结果:Local Time is 2.001146, Global Time is 10.809598
\\

3. 编程实现。

\lstinline{MPI_Bcast、MPI_Scatter、MPI_Gather、MPI_Allgather、MPI_Alltoall}是典型的以通信为主的MPI集合通讯语句。请选择以上的任意一条语句,分别采用
\begin{enumerate}[label=(\alph*)]
    \item \lstinline{MPI_Send、MPI_Recv}
    \item \lstinline{MPI_Isend、MPI_Irecv}
\end{enumerate}
两种方法实现其功能,并且使用\lstinline{MPI_Wtime}计时并做性能分析,并且把a、b两种实现和MPI通信库中标准的集合操作语句比较。

以上通讯语句的定义如下:

\begin{lstlisting}
    int MPI_Bcast(void *buffer, int count, MPI_Datatype datatype, int root, MPI_Comm comm)
    int MPI_Gather(const void *sendbuf, int sendcount, MPI_Datatype sendtype, void *recvbuf,int recvcount, MPI_Datatype recvtype, int root, MPI_Comm comm)
    int MPI_Gatherv(const void *sendbuf, int sendcount, MPI_Datatype sendtype, void *recvbuf, const int *recvcounts, const int *displs, MPI_Datatype recvtype, int root, MPI_Comm comm)
    int MPI_Scatter(const void *sendbuf, int sendcount, MPI_Datatype sendtype, void *recvbuf, int recvcount, MPI_Datatype recvtype, int root, MPI_Comm comm)
    int MPI_Allgather(const void *sendbuf, int sendcount, MPI_Datatype sendtype, void *recvbuf, int recvcount, MPI_Datatype recvtype, MPI_Comm comm)
    int MPI_Alltoall(const void *sendbuf, int sendcount, MPI_Datatype sendtype, void *recvbuf, int recvcount, MPI_Datatype recvtype, MPI_Comm comm)
\end{lstlisting}


例如:\lstinline{MPI_Bcast}的可能实现如下:

mybcast.c

\begin{lstlisting}
#include "mpi.h"
#include <stdio.h>
#include <assert.h>
#define BUF_SZ 64
int main(int argc, char* argv[])
{
    int rank, size, i;
    double t_start, t_end, l_time, g_time;
    int data[BUF_SZ];
    MPI_Init(&argc,&argv);
    MPI_Comm_rank(MPI_COMM_WORLD, &rank);
    MPI_Comm_size(MPI_COMM_WORLD, &size);

    if (!rank) 
    for (i=0; i<BUF_SZ; i++)
        data[i] = i+1;

    t_start = MPI_Wtime();
    if (!rank) {
    for (i=1; i<size; i++)
        MPI_Send(data, BUF_SZ, MPI_INT, i, 123, MPI_COMM_WORLD);
    } else 
    MPI_Recv(data, BUF_SZ, MPI_INT, 0, 123, MPI_COMM_WORLD, MPI_STATUS_IGNORE);
    t_end= MPI_Wtime();

    for (i=0; i<BUF_SZ; i++)
    assert(data[i] == i+1);

    l_time = t_end - t_start;
    g_time = 0;
    MPI_Reduce(&l_time, &g_time, 1, MPI_DOUBLE, MPI_MAX, 0, MPI_COMM_WORLD);
    if (!rank) 
    printf("Local Time is %f, Global Time is %f\n", l_time, g_time);
    MPI_Finalize();

    return 0;
}
\end{lstlisting}

编译与运行:
\begin{lstlisting}
mpicc mybcast.c –o mybcast
yhrun –n 32 ./mybcast
\end{lstlisting}

选择合适的实验方法,降低Wtime计时的实验误差。
\\

说明:
选择\lstinline{MPI_Scatter},分别使用阻塞/非阻塞发送接收方法对其
进行实现。其中,阻塞方式函数命名为\lstinline{MPI_Scatter_cz},
非阻塞方式函数命名为\lstinline{MPI_Scatter_czi},详见源程序
ex3.c。

为了评估以上两种实现与MPI通信库中的实现,对这三种实现进行性能评
估,使用\lstinline{MPI_Wtime}计时进行性能分析,为了降低计时误差
,在实验中执行Scatter操作N次取和,且最终结果重复三次取平均获
得。实验结果如表\ref{labelex3}。
\\

从结果可以看出,阻塞方式实现已经接近 OpenMPI 标准库效率,非阻塞
由于本实验没有过多的计算内容,无法发挥非阻塞方式的优势,额外的
\lstinline{MPI_Wait}函数判断占用了较多的时间,导致性能弱于阻塞
方式。

\begin{table}[]
    \centering
    \caption{三种实现执行时间对比}
    \label{labelex3}
    \begin{tabular}{@{}lllll@{}}
        \toprule
        N=1000000 &          &          &          & mean \\ \midrule
        阻塞   & 1.033502 & 1.029479 & 1.051039 &1.038     \\
        非阻塞  & 1.438418 & 1.402808 & 1.329164 &1.390      \\
        OpenMPI & 1.021237 & 1.020868 & 1.011095 &1.012      \\ \bottomrule
    \end{tabular}
\end{table}

\end{document}
