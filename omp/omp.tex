% 使用 UTF-8 编码 
\documentclass{ctexart}
\usepackage{listings}
\usepackage{enumitem}
\usepackage{courier}
\usepackage{booktabs}
\setCJKmainfont{SimSun}
\setlength\parindent{0em}

\lstset{basicstyle=\footnotesize\ttfamily,breaklines=true}

\author{程振}
\title{高级并行程序作业——OpenMP}

\begin{document}
\maketitle

1. 编写求$\pi$的程序,并且用OpenMP并行,分析其性能。

程序编译的命令为:

\qquad GNU编译器:\lstinline {gcc -fopenmp pi.c -o pi}

\qquad INTEL编译器:\lstinline {icc -openmp pi.c -o pi}

程序运行的命令为:

\qquad \lstinline {OMP_NUM_THREADS=16 yhrun -N 1 -p free ./pi}
\\

{\heiti 解答:}
使用 Fortran 编程语言,利用 Simpson's rule 计算
\lstset{language=C}
\begin{equation}
     \pi= \int_0^1 \frac{1}{1+x^2} dx
\end{equation}
\qquad 为了比较普通串行程序与OpenMP并行程序的性能差异,编写了对应的两个程序,分别为\lstinline{ex1.f90}与\lstinline{ex1_omp.f90}.
性能分析主要考虑程序的执行时间,通过调用secnds()函数获得。详见源程序。

\qquad 对于不同的分割区间宽度$w$,不同程序(线程)的执行时间见表 ~\ref{tab:wvstime}。由于secnds()函数的精度有限,当计算中选取的分割区间较小时($<10^5$),程序执行时间过短,无法评估其性能,对应于表中的时间数值为0,但这一点并不妨碍后面的性能分析。

\qquad 比较串行程序与单线程OpenMP并行程序的计算结果,可以发现二者数值十分接近,这是符合预期的。

\qquad 比较不同线程的OpenMP并行程序,可以看到多线程可以起到明显的加速效果。对于一定规模的问题,在线程数较小时,增加一倍线程,计算时间对应减半,加速效果十分显著,随着线程的继续增多,由于问题规模保持不变,并行计算的开销比重开始增大,加速的效果开始减弱。
\\



\begin{table}[]
    \caption{不同分割区间宽度不同程序(线程)的执行时间(单位:s)}
\label{tab:wvstime}
\centering
\begin{tabular}{@{}lllll@{}}
    \toprule
   & $w=10^3$ & $w=10^5$ & $w=10^7$ & $w=10^9$ \\ \midrule
串行   &0.00000000          &0.00200000          &0.17297401          & 17.08593750          \\
单线程 &0.00000000          &0.00200000          &0.18797100          &17.04140854          \\
4线程  &0.00000000          &0.00000000          & 0.05468750          &4.55468750        \\
8线程  &0.00000000          &0.00000000          &0.03125000          &2.40625000          \\
16线程 &0.00000000          &0.00000000          & 0.01562500          &1.24218750          \\
32线程 &0.00000000          &0.00000000          &0.02343750          &0.97656250 \\
64线程 &0.00000000          &0.00000000          &0.01562500          &0.89062500 \\ \bottomrule
\end{tabular}
\end{table}

2. 考虑下面的循环:

\begin{lstlisting}
a[0] = 0;
for (i=0; i<n; i++)
	a[i+1] = a[i] + i;
\end{lstlisting}

这个循环中有循环依赖,不能直接使用OpenMP并行,请找一种方法消除循环依赖,并且并行化该循环。
\\

{\heiti 解答:}
将数组$a[]$视为数列,由递推公式
$$
a[i+1] = a[i]+i, i>=0
$$

可得其通项公式:
$$
a[i+1] = \frac{i(i+1)}{2}, i>=0
$$
利用通项公式可以避免递推公式中存在的循环依赖,据此编写其OpenMP并行程序,ex2.f90.
\\

3. 解线性方程组时,经常使用高斯消元法,高斯消元法吧一个n*n的矩阵转换成一个上三角矩阵(矩阵对角线一下元素为0),基本操作为:
\begin{enumerate}
\item 两行相加
\item 两行互换
\item 一行乘以非0常数
\end{enumerate}

例如线性方程组

\begin{equation}
  \left\{
   \begin{array}{c}
   2x_0 - 3x_1 = 3  \\
   4x_0 - 5x_1 + x_2 = 7  \\
   2x_0 - x_1 + -3x_2 = 5  
   \end{array}
  \right.
\end{equation}

可以整理成上三角矩阵
\begin{equation}
  \left\{
   \begin{array}{c}
   2x_0 - 3x_1 = 3  \\
   5x_1 + x_2 = 1  \\
   -5x_2 = 5  
   \end{array}
  \right.
\end{equation}
以上三角矩阵就可以求解$x_0$, $x_1$, $x_2$.

通常高斯消元法有两个版本“行优先”和“列优先”版本,假设方程组为$Ax=b$,“行优先”版本是:
\begin{lstlisting}
for (row=n-1; row>=0; row--) {
	x[row] = b[row];
	for (col= row+1; col<n; col++)
		x[row] -= A[row][col]*x[col];
	x[row] /= A[row][row];
}
\end{lstlisting}
列优先版本如下:
\begin{lstlisting}
for (row=0; row<n; row++)
    x[row] = b[row];
for (col=n-1; col>=0; col--) {
    x[col] /= A[col][col];
    for (row=0; row<col; row++)
        x[row] -= A[row][col]*x[col];
}
\end{lstlisting}

两种方法均有两层循环,请判断并且说明原因:
\begin{enumerate}[label=(\alph*)]
\item “行优先”算法外层循环可否openmp并行化;
\item “行优先”算法内层循环可否openmp并行化;
\item “列优先”算法外层循环可否openmp并行化;
\item “列优先”算法内层循环可否openmp并行化;
\item 使用可行的并行化方法编写一个OpenMP并行的高斯消去法程序。注:可能用到Single语句。
\item 用Schedule语句修改并行循环,分别假设矩阵有10000变量,20000变量,40000变量,选择各自最优的调度策略。
\end{enumerate}

\end{document}
