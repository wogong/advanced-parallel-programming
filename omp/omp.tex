% 使用 UTF-8 编码 
\documentclass{ctexart}
\usepackage{listings}
\usepackage{enumitem}
\usepackage{courier}
\usepackage{booktabs}
\setCJKmainfont{SimSun}
\setlength\parindent{0em}

\lstset{basicstyle=\footnotesize\ttfamily,breaklines=true}

\author{程振}
\title{高级并行程序作业——OpenMP}

\begin{document}
\maketitle

1. 编写求$\pi$的程序,并且用OpenMP并行,分析其性能。

程序编译的命令为:

\qquad GNU编译器:\lstinline {gcc -fopenmp pi.c -o pi}

\qquad INTEL编译器:\lstinline {icc -openmp pi.c -o pi}

程序运行的命令为:

\qquad \lstinline {OMP_NUM_THREADS=16 yhrun -N 1 -p free ./pi}
\\

{\heiti 解答:}
使用 Fortran 编程语言,利用 Simpson's rule 计算
\lstset{language=C}
\begin{equation}
     \pi= \int_0^1 \frac{1}{1+x^2} dx
\end{equation}
\qquad 为了比较普通串行程序与OpenMP并行程序的性能差异,编写了对应的两个程序,分别为\lstinline{ex1.f90}与\lstinline{ex1_omp.f90}.
性能分析主要考虑程序的执行时间,通过调用secnds()函数获得。详见源程序。

\qquad 对于不同的分割区间宽度$w$,不同程序(线程)的执行时间见表 ~\ref{tab:wvstime}。由于secnds()函数的精度有限,当计算中选取的分割区间较小时($<10^5$),程序执行时间过短,无法评估其性能,对应于表中的时间数值为0,但这一点并不妨碍后面的性能分析。

\qquad 比较串行程序与单线程OpenMP并行程序的计算结果,可以发现二者数值十分接近,这是符合预期的。

\qquad 比较不同线程的OpenMP并行程序,可以看到多线程可以起到明显的加速效果。对于一定规模的问题,在线程数较小时,增加一倍线程,计算时间对应减半,加速效果十分显著,随着线程的继续增多,由于问题规模保持不变,并行计算的开销比重开始增大,加速的效果开始减弱。
\\



\begin{table}[]
    \caption{不同分割区间宽度不同程序(线程)的执行时间(单位:s)}
\label{tab:wvstime}
\centering
\begin{tabular}{@{}lllll@{}}
    \toprule
   & $w=10^3$ & $w=10^5$ & $w=10^7$ & $w=10^9$ \\ \midrule
串行   &0.00000000          &0.00200000          &0.17297401          & 17.08593750          \\
单线程 &0.00000000          &0.00200000          &0.18797100          &17.04140854          \\
4线程  &0.00000000          &0.00000000          & 0.05468750          &4.55468750        \\
8线程  &0.00000000          &0.00000000          &0.03125000          &2.40625000          \\
16线程 &0.00000000          &0.00000000          & 0.01562500          &1.24218750          \\
32线程 &0.00000000          &0.00000000          &0.02343750          &0.97656250 \\
64线程 &0.00000000          &0.00000000          &0.01562500          &0.89062500 \\ \bottomrule
\end{tabular}
\end{table}

2. 考虑下面的循环:

\begin{lstlisting}
a[0] = 0;
for (i=0; i<n; i++)
	a[i+1] = a[i] + i;
\end{lstlisting}

这个循环中有循环依赖,不能直接使用OpenMP并行,请找一种方法消除循环依赖,并且并行化该循环。
\\

{\heiti 解答:}
如果可以使用通项公式,则参见方法1,否则方法2。

{\heiti 方法1}

将数组$a[]$视为数列,由递推公式
$$
a[i+1] = a[i]+i, i>=0
$$

可得其通项公式:
$$
a[i+1] = \frac{i(i+1)}{2}, i>=0
$$
利用通项公式可以避免递推公式中存在的循环依赖,据此编写其OpenMP并行程序,ex2.f90.

{\heiti 方法2}

预先计算出a[9]的数值,将程序分解为两块可并行的程序块,解决循环依赖,可两个线程执行。
如果需要更多线程执行可以采用类似的方法,预先计算出特定值,将程序分解为多个可并行执行
的程序块。
据此编写OpenMP并行程序,\lstinline{ex2_1.f90}
\\


3. 解线性方程组时,经常使用高斯消元法,高斯消元法吧一个n*n的矩阵转换成一个上三角矩阵(矩阵对角线一下元素为0),基本操作为:
\begin{enumerate}
\item 两行相加
\item 两行互换
\item 一行乘以非0常数
\end{enumerate}

例如线性方程组

\begin{equation}
  \left\{
   \begin{array}{c}
   2x_0 - 3x_1 = 3  \\
   4x_0 - 5x_1 + x_2 = 7  \\
   2x_0 - x_1-3x_2 = 5  
   \end{array}
  \right.
\end{equation}

可以整理成上三角矩阵
\begin{equation}
  \left\{
   \begin{array}{c}
   2x_0 - 3x_1 = 3  \\
   x_1 + x_2 = 1  \\
   -5x_2 = 0  
   \end{array}
  \right.
\end{equation}
以上三角矩阵就可以求解$x_0$, $x_1$, $x_2$.

通常高斯消元法有两个版本“行优先”和“列优先”版本,假设方程组为$Ax=b$,“行优先”版本是:
\begin{lstlisting}
for (row=n-1; row>=0; row--) {
	x[row] = b[row];
	for (col= row+1; col<n; col++)
		x[row] -= A[row][col]*x[col];
	x[row] /= A[row][row];
}
\end{lstlisting}
列优先版本如下:
\begin{lstlisting}
for (row=0; row<n; row++)
    x[row] = b[row];
for (col=n-1; col>=0; col--) {
    x[col] /= A[col][col];
    for (row=0; row<col; row++)
        x[row] -= A[row][col]*x[col];
}
\end{lstlisting}

两种方法均有两层循环,请判断并且说明原因:
\begin{enumerate}[label=(\alph*)]
\item “行优先”算法外层循环可否openmp并行化;
\item “行优先”算法内层循环可否openmp并行化;
\item “列优先”算法外层循环可否openmp并行化;
\item “列优先”算法内层循环可否openmp并行化;
\item 使用可行的并行化方法编写一个OpenMP并行的高斯消去法程序。注:可能用到Single语句。
\item 用Schedule语句修改并行循环,分别假设矩阵有10000变量,20000变量,40000变量,选择各自最优的调度策略。
\end{enumerate}

{\heiti 解答:}
对于高斯消去法,首先将线性方程组系数矩阵转化成上三角矩阵,对于上三角矩阵形式的系数矩阵$u_{nn}$,可以
直接求得方程的根:
$$
x_{i} = \frac{b_i-\sum_{j=i+1}^{n} u_{ij} x_{j} }{u_{ii}}, i=n,...,1
$$
本题的源程序为\lstinline{ex3.c},函数\lstinline{gaussr}和函数
\lstinline{gaussc}分别代表行优先与列优先的算法。

\begin{enumerate}[label=(\alph*)]
\item “行优先”算法外层循环不可openmp并行化,外层循环是对每一行
    进行回代求解,由于回代是从最后一行开始,之后的每一行都
    需要上一行的回代结果,所以各行间即外层循环间存在数据依赖;
\item “行优先”算法内层循环可以openmp并行化,通过数学公式可以看
    出,内层循环主要为求和,没有循环间数据依赖;
\item “列优先”算法外层循环不可openmp并行化,外层循环从最后一列
    开始,对每列进行回代,循环间存在数据依赖。
\item “列优先”算法内层循环可以openmp并行化,在获得此列之后x值的
    前提下,内层循环对每行分别进行求和,不存在数据依赖。
\item  选择行优先算法,使得内层循环并行,对应的OpenMP程序见
    源程序中的
    \lstinline{gaussc_omp}函数。
\item 为了分析的简便性,下面OpenMP程序在实验过程中均使用4线程。
    对于10000、20000与40000变量的线性方程组,其采用调度方式为
    static、dynamic和guided的运行时间见以下三个表格(所有调度
    方式参数均采用默认值)。

    % Please add the following required packages to your document preamble:
    % \usepackage{booktabs}
        \begin{table}[]
            \centering
            \caption{N=10000,unit:$\mu s$}
            \label{my-label}
            \begin{tabular}{@{}lllll@{}}
                \toprule
                N=10000 &         &         &         & average  \\ \midrule
                static  & 375344  & 372042  & 369511  & 372299   \\
                dynamic & 4621399 & 4903032 & 4926575 & 4817002  \\
                guided  & 157510  & 162388  & 163820  & 1612393 \\ \bottomrule
            \end{tabular}
        \end{table}

        % Please add the following required packages to your document preamble:
        % \usepackage{booktabs}
        \begin{table}[]
            \centering
            \caption{N=20000,unit:$\mu s$}
            \label{my-label}
            \begin{tabular}{@{}lllll@{}}
                \toprule
                N=20000 &          &          &          & average  \\ \midrule
                static  & 1277954  & 1294309  & 1288558  & 1286940  \\
                dynamic & 18785700 & 19571417 & 19176848 & 19177988 \\
                guided  & 455690   & 471719   & 460647   & 462685   \\ \bottomrule
            \end{tabular}
        \end{table}


        % Please add the following required packages to your document preamble:
        % \usepackage{booktabs}
        \begin{table}[]
            \centering
            \caption{N=40000,unit:$\mu s$}
            \label{my-label}
            \begin{tabular}{@{}lllll@{}}
                \toprule
                N=40000 &          &          &          & average  \\ \midrule
                static  & 5004907  & 4862696  & 5274263  & 5047289  \\
                dynamic & 72441867 & 78517320 & 72950605 & 74636597 \\
                guided  & 1514067  & 1478113  & 1473502  & 1488561  \\ \bottomrule
            \end{tabular}
        \end{table}

        
        通过表格数据可以发现,对于不同规模的变量,调度方式优劣
        排序为guided>static>dynamic.
        
        以上为使用默认参数下的结果,对于size的选择也会严重影响
        不同调度方式的效率,本例中,dynamic选择较大的size
        (例如100)会使得其效率优于默认的static调度方式,但是依旧
        不如guided调度方式。

        当对于问题的特性有较清楚的了解时,可以据此选择合适的
        调度方式和参数。否则,建议选用guided调度方式,在默认
        情况下也能得到较好的性能。

\end{enumerate}


\end{document}
